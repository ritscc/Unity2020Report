\documentclass[a4paper]{jarticle}

\usepackage[top=2cm,bottom=2cm,left=2cm,right=2cm]{geometry}

 \usepackage{url}

\renewcommand\thefootnote{*\arabic{footnote}}

\newcommand{\resp}[1]{\begin{flushright}文責: #1\end{flushright}~\\ }

\title{Unity班 活動報告書}
\author{立命館コンピュータクラブ\\2021年度プロジェクト活動}
\date{2021年2月8日}

\begin{document}

  \maketitle
  \begin{center}
    % 名前 \footnote{所属学部 所属学科 (あれば所属コース) 回生}というように記述
    堀越 俊行\footnote{情報理工学部 情報理工学科 システムアーキテクトコース 三回生}
        新藤 尚輝\footnote{情報理工学部 情報理工学科 実世界情報コース 二回生}
        宇佐 基史\footnote{理工学部 ロボティクス科 二回生}

    原 佑馬\footnote{情報理工学部 情報理工学科 システムアーキテクトコース 三回生}
      北村 優奈\footnote{情報理工学部 情報理工学科 システムアーキテクトコース 三回生}
        中山 凌一\footnote{情報理工学部 情報理工学科 システムアーキテクトコース 三回生}

    服部 瑠斗\footnote{情報理工学部 情報理工学科 知能情報コース 三回生}
      岡本 陽太\footnote{情報理工学部 情報理工学科 実世界情報コース 三回生}
        桐井 優実\footnote{情報理工学部 情報理工学科 システムアーキテクトコース 二回生}

    中川 拓海\footnote{情報理工学部 情報理工学科 セキュリティ・ネットワークコース 三回生}
        伊藤 佑\footnote{}
        山本 京介\footnote{情報理工学部 情報理工学科 システムアーキテクトコース 一回生}

  \end{center}

  \newpage

  \tableofcontents

  \newpage

  \section{Unity班について}
    \resp{宇佐 基史}
    
  \section{活動について}
    \resp{北村 優奈,原 佑馬}

 \subsection{個人の作成物紹介}
   ここでは,個人の作成物の紹介を行います.

   \subsubsection{}
        \resp{氏名}

 \subsection{年間の活動の上での問題点}
      \resp{山本 京介}
     
    \subsection{年間の活動で得られた知見}
      \resp{桐井 優実}

    \subsection{班総括}
      \resp{新藤 尚輝}
  
\end{document}
